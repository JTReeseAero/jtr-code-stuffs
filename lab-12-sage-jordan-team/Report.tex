\documentclass{article} \usepackage[utf8]{inputenc} \title{Lab 12 Report} 
\author{Sage Salmon and Jordan Reese } \date{April 15 2018} 
\usepackage{listings} \usepackage{color} 
\definecolor{dkgreen}{rgb}{0,0.6,0} \definecolor{gray}{rgb}{0.5,0.5,0.5} 
\definecolor{mauve}{rgb}{0.58,0,0.82} \lstset{frame=tb,
  language=Java,
  aboveskip=3mm,
  belowskip=3mm,
  showstringspaces=false,
  columns=flexible,
  basicstyle={\small\ttfamily},
  numbers=none,
  numberstyle=\tiny\color{gray},
  keywordstyle=\color{blue},
  commentstyle=\color{dkgreen},
  stringstyle=\color{mauve},
  breaklines=true,
  breakatwhitespace=true,
  tabsize=3
}
\begin{document} \maketitle \section*{Introduction} The main task in this 
lab was running a simulation of two characters who needed to pass 
different challenges in order to move around a room. When run, the code 
chose several of the challenges when run and if passed, the characters 
would be allowed to move in the specified direction. If the challenges did 
not pass the characters would not move. \section*{Challenges} There was a 
list of challenges that we were required to write code so each challenge 
would pass. These challenges are listed below: \begin{itemize}
    \item \textbf{Calculation}: Given a string of a mathematical 
expression, return the integer value
    \item \textbf{Sub-String}: Given two strings, return a pointer to the 
beginning of the first occurrence of the specified string
    \item \textbf{Range}: Given two integers, return the range between 
them as an integer
    \item \textbf{Mean}: Given an array of floats and the number of floats 
in the array, return the arithmetic mean as a float
    \item \textbf{Minimum}: Given an array of floats and the number of 
floats in the array, return the minimum value as a float
    \item \textbf{Maximum}: Given an array of floats and the number of 
floats in the array, return the maximum value as a float
    \item \textbf{Reverse}: Given a string, return the string in reverse
    \item \textbf{Find}: Given a string and a char, return the integer 
index of the position of the first occurrence of the char
    \item \textbf{Tokenize}: Given a string, tokenize it by whitespace and 
return a pointer to a malloced 1-D array of the token strings. 
\end{itemize} Whether or not these challenges passed when called was the 
determining factor in how the characters moved. Many of these challenges 
were not too complicated to understand or to code. Several of them were a 
means of doing simple mathematical operations and just needed the syntax 
done properly. On the other hand, a few of the challenges, such as 
\textbf{Calculation} and \textbf{Tokenize}, were more challenging and 
required more work to figure out.\newline \indent The next hardest part of 
the challenges was getting all of the answers passed to the proper 
location so the simulator could verify our answer. The simulator gave us a 
void pointer that we needed to cast our answer to. There were a couple 
ways you could do this. You could directly cast the answer using some 
pointer notation, or you can do what we did and make a union. Our union 
contained all of the data types needed for the challenge answers. We 
pointed the void pointer at the union and then never had to worry about 
the void pointer again. We only had to make sure the data type our 
function returned matched the one needed in the union. \section*{Handling 
the challenges} We created a function that passes all of the returns 
values from the challenges function to the void pointer and makes sure the 
simulator gets the correct answers. Below is the handle function, 
function. It is just a series of else if statement that check to see with 
function was called. This function makes it easy to determine where an 
error would be if the simulator doesn't receive our answer. 
\begin{lstlisting} int Handle_Challenge(struct Round round){
  printf("\n\n");
  union Challenge_Answer *ans = round.ans_ptr;
  if (round.chal_typ == calc){
    // printf("Did I get here?\n");
    ans->i = Calc(round.chal_dat.calc);
    // *(int *)round.ans_ptr = Calc(round.chal_dat.calc);
    //ans->i = 5;
  }
  else if (round.chal_typ == substr){
    ans->str = Sub_Str(round.chal_dat.substr.s1, 
round.chal_dat.substr.s2);
  }
  else if (round.chal_typ == range){
    ans->i = Range(round.chal_dat.range.a, round.chal_dat.range.b);
  }
  else if (round.chal_typ == mean){
    ans->f = Mean(round.chal_dat.minmaxmean.nums, 
round.chal_dat.minmaxmean.datalen);
  }
  else if (round.chal_typ == min){
    ans->f = Min(round.chal_dat.minmaxmean.nums, 
round.chal_dat.minmaxmean.datalen);
  }
  else if (round.chal_typ == max){
    ans->f = Max(round.chal_dat.minmaxmean.nums, 
round.chal_dat.minmaxmean.datalen);
  }
  else if (round.chal_typ == rev){
    Reverse(round.chal_dat.rev, ans->arr);
  }
  else if (round.chal_typ == find){
    ans->i = Find(round.chal_dat.find.s1, round.chal_dat.find.key);
  }
  else if (round.chal_typ == token){
   ans->str_array = Token(round.chal_dat.token);
}
  return 0;
}
\end{lstlisting} \section*{Movement} The movement was fairly simple when 
it came to what to code. There was a vision matrix with ones and zeros 
which distinguished yes from no and there were cardinal directions given 
with the rows. The code we wrote was four if statements that checked to 
see which movement direction passed and then moved the character in that 
direction. The movement code is below. \begin{lstlisting} int move(struct 
Round round){
  int i = 0;
  int j = 0;
  for(i = 0; i < 3; i++){
    for(j = 0; j < 3; j++){
      printf("%d ", round.vision[i][j]);
   
    }
    printf("\n");
  }
  if(round.vision[0][1] != true){
    return 1;
  }
  else if(round.vision[1][2] != true){
    return 2;
  }
  else if(round.vision[2][1] != true){
    return 3;
  }
  else if(round.vision[1][0] != true){
    return 4;
  }
  return 0;
}
\end{lstlisting} This is not even close to the most efficient way to move 
around the map, but it does guarantee that the character always moves 
unless it is completely trapped. This algorithm does have issues when the 
character starts in a corner. We didn't spend too much time on the 
movement because in our minds it wasn't the main focus of the lab. 
\section*{Issues} There were a lot of problems that we ran across when 
working through the lab requirements. The biggest issue we kept running 
into was a problem with the lab given code. We were working through the 
challenges and writing additional code to check to make sure the code 
returned the correct values, however, the challenges still failed in the 
simulation for most of the challenges. This issue was not resolved for 
several days, which gave us grief when we were wanting to move forward 
with the lab instructions. We also had issues interpreting some of the lab 
manual instructions. Some of the requirements were very unclear and worded 
poorly. There needs to be more elaborated on some of the challenges and 
how exactly they are supposed to be done or what they are looking for. 
\newline \indent The gnuplot was probably the worst part of this lab. 
Using the program had a really high learning curve, and we still couldn't 
figure out how to properly print out a heat map. The map as it sits will 
roughly show where the ones are located. The gradient prevents any super 
solid colors from popping out. But we did figure out how to uses the 
gnuplot inside the c code. Our code pulls a script from the directory to 
make life a little easier. \newline \indent Another problem we had was 
writing the tokenize function. The built in C function strtok that can 
tokenize has a problem. It changes the string passed into it. We had to 
make several copies of the original string so we didn't alter the pointer. 
\section*{Complexity} This code is not very complex. The code varies in 
complexity based on the challenges passed to it. If we are given some of 
the easier functions our Big O is O(n) and our complexity goes up to 
O(n)$^2$. We have our loops buggin out if there are error, and if any of 
the files don't get opened properly. Based on what the simulator sends us 
then we should be able to compensate for an errors. \section*{Conclusion} 
This lab seemed to be a fair amount of work but within the ability of 
students in this course. There were a lot of functions, but most of them 
were very simple and did not take much time or effort when creating the 
code. After getting clarification and understanding what the lab manual 
wanted, the coding was not overly complicated. The hardest part of this 
lab was trying to work through it while the simulation was not working. We 
had to delay some of our work because we couldn't debug without knowing if 
our challenges passed or not. All in all, the exercises in this lab 
exercised the skills we have been learning throughout the semester and 
showed a physical impact that our code had on a simulation. We were able 
to see changes in something that didn't just involve lines of code or 
numbers on a screen, which made it easier to realize the impacts knowing 
good coding practice can make. \end{document}
