\documentclass{article} \usepackage[utf8]{inputenc} \title{AERE 361 Lab 9} \author{Jordan Reese} \date{March, 23 2018} \usepackage{listings} \usepackage{color} 
\definecolor{dkgreen}{rgb}{0,0.6,0} \definecolor{gray}{rgb}{0.5,0.5,0.5} \definecolor{mauve}{rgb}{0.58,0,0.82} \lstset{frame=tb,
  language=Java,
  aboveskip=3mm,
  belowskip=3mm,
  showstringspaces=false,
  columns=flexible,
  basicstyle={\small\ttfamily},
  numbers=none,
  numberstyle=\tiny\color{gray},
  keywordstyle=\color{blue},
  commentstyle=\color{dkgreen},
  stringstyle=\color{mauve},
  breaklines=true,
  breakatwhitespace=true,
  tabsize=3
}
\begin{document} \maketitle \section*{Exercise 2} The ranges of the unsigned systems. \begin{itemize}
    \item 8-bit = 0 to 255
    \item 16-bit = 0 to 65, 535
    \item 32-bit = 0 to 4,294, 967, 295
    \item 64-bit = 0 to 18, 446, 744, 073, 709, 551, 615 \end{itemize} The ranges of the signed systems. \begin{itemize}
    \item 8-bit = -128 to 127
    \item 16-bit = -32,768 to 32,767
    \item 32-bit = -2,147,483,648 to 2,147,483,647
    \item 64-bit = -9,223,372,036,854,775,808 to 9,223,372,036,854,775,807 \end{itemize} The value of the 8-bit unsigned representation \begin{itemize}
    \item 88 = 0101 1000
    \item 0 = 0000 0000
    \item 1 = 0000 0001
    \item 127 = 0111 1111
    \item 255 = 1111 1111 \end{itemize} The value of the 2's complement 8-bit representation \begin{itemize}
    \item +88 = 0101 1000
    \item -88 = 1010 1000
    \item -1 = 1111 1111
    \item 0 = 0000 0000 and 1000 0000
    \item 1 = 0000 0001
    \item -128 = 1000 0000
    \item 127 = 0111 1111 \end{itemize} The value of the 8-bit signed representation \begin{itemize}
    \item 88 = 0101 1000
    \item -88 = 1101 1000
    \item -1 = 1000 0001
    \item 0 = 0000 0000
    \item 1 = 0000 0001
    \item -127 = 0111 1111
    \item 127 = 1111 1111 \end{itemize} The value of the 1's complement 8-bit representation \begin{itemize}
    \item 88 = 0101 1000
    \item -88 = 1010 0111
    \item -1 = 1111 1110
    \item 0 = 0000 0000 and 1111 1111
    \item 1 = 0000 0001
    \item -127 = 1000 0000
    \item 127 = 0111 1111 \end{itemize} \section{Exercise 3} The largest and smallest normalized positive numbers \begin{itemize}
    \item Largest in binary is S$=$0 E$=$1111 1110 F$=$23 ones\newline Factoring in the leading point and the bias we have 1.(23 ones) in binary converts to 
about $ 1.9999 * 2^{127} = 3.4028235 * 10^{38} $\newline It is the same for negative numbers but S = 1.
    \item Smallest in binary is S = 0 E = 0000 0001 F = 23 zeros\newline Factoring in the leading point and the bias we have 1.(23 zeros) in binary converts to 
about $1.0 * 2^{-126} = 1.17549435 * 10^{-38} $\newline It is the same for negative numbers but S = 1.\newline
    \item Denormalized is the process but our E is only zeros.\newline The largest is S = 0 E = 0000 0000 and F = 23 ones
    The equation is $(1-2^{-23}) * 2^{-126} = 1.1754942 * 10^{-38}$ and the negative is the same but S = 1.
    \item The smallest is S = 0 E = 0000 0000 F = 23 zeros\newline
    The equation is $ 1 * 2^{-23} * 2^{-126} = 2^{-149} = 1.4 * 10^{-45}$ and the negative is the same but S = 1. \end{itemize} \section*{Exercise 4} Minimum 
value for normalized 64-bit: 2.22507 E-308 \newline Maximum value for normalized 64-bit: 1.797693 E+308 \newline Minimum value for denormalized 64-bit: 4.9 
E-324 \newline Maximum value for denormalized 64-bit: 4.4501477 E-308 \section{Exercise 5} I want to start by saying this section is very gross. I had to read 
multiple documents and chat with a professional programmer in order to get this program in working order, and for it to work against most inputs. One of the 
things I struggle with most in this lab was trying to understand what the lab actually wanted us to do. We needed to create our own data type. I set up struct. 
This is shown below \begin{lstlisting} struct MyFloat{
  char *mantissa;
  char *expo;
  int sign;
  int expobits;
  int mantbits;
};
\end{lstlisting} The struct will be passed through functions in order to properly initialize the struct and set values to it. The other thing we had to do for 
this lab was dynamically allocate memory to our values, so there is a function used to allocate the struct. The last function simply prints the values in my 
struct. Setting up all of these functions isn't any less typing, but it does make my main code really really pretty. I will list the functions below and talk 
briefly about them, because I know you understand what they are saying. \newline \newline Initialize MyFloat: \begin{lstlisting} int init_myfloat(struct MyFloat 
*in, int expobits, int mantbits){
  in->sign = 0;
  in->mantissa = malloc(mantbits * sizeof(char));
  in->expo = malloc(expobits * sizeof(char));
  in->expobits = expobits;
  in->mantbits = mantbits;
  return 0;
}
\end{lstlisting} This is the function to initialize my struct. I set values to zero and allocate the memory here. Also when this function is called we are after 
call in main for the bits of exponent and mantissa, so I store those in there respective locations as well.\newline \newline Freeing Memory: \begin{lstlisting} 
int free_myfloat(struct MyFloat *in){
  free(in->mantissa);
  free(in->expo);
  in->mantissa = NULL;
  in->expo = NULL;
  return 0;
}
\end{lstlisting} This was really easy to do. All I used was free the allocated memory and set some numbers to null so they don't start doing silly 
things.\newline \newline Printing MyFloat: \begin{lstlisting} int print_myfloat(struct MyFloat *in){
  int i = 0;
  printf("Your floating point bit representation is: %d ",in->sign);
  for (i = 0; i < in->expobits; i++){
    printf("%d", in->expo[i]);
  }
  printf(" ");
  for (i = 0; i < in->mantbits; i++){
    printf("%d", in->mantissa[i]);
  }
  printf("\n");
  return 0;
}
\end{lstlisting} All this code did was set up the printing setup for sign, exponent, and mantissa. Printing in each of the fields was also easy because of how I 
set up my struct. I just looped over and index and printed out the values. \newline \newline Setting Values in MyFloat (aka Hell): \begin{lstlisting} int 
set_myfloat(struct MyFloat *in,int sign, int whole, double dec){
  in->sign = sign;
  char buffer[1000];
  int shift = 0;
  decimaltobinary(whole,dec,buffer);
  printf("%s\n", buffer);
  
  char *p;
  p = strchr(buffer, '.');
  int numleft = p - buffer;
  if (1 == numleft){
    shift = 0;
  }
  else if(0 == numleft) {
    char *one;
    one = strchr(buffer,'1');
    if (NULL == one){
      in->sign = 0;
      return 0;
    }
    else{
      int numzero = one - (buffer + 1);
      shift = -(numzero + 1);
    }
  }
  else{
    shift = numleft-1;
  }
  //bias = 2^(expobits-1)-1
  int bias = 0;
  bias = pow(2,in->expobits-1) - 1;
  int exponent = shift + bias;
  char expbuff[100];
  decimaltobinary(exponent, 0, expbuff);
  printf("%s\n", expbuff);
  
  *strchr(expbuff,'.') = 0;
  printf("%s\n", expbuff);
  int index = 0;
  int lead = in->expobits - strlen(expbuff);
  if (lead >= 0){
    for(index = 0; index < strlen(expbuff); index++){
      in->expo[index+lead] = ('0' == expbuff[index]) ? 0 : 1;
    }
  }
  else{
    //too many bits in expbuff than can fit in expo
    char *h = expbuff - lead;
    for (index = 0; index < strlen(h); index++){
      in->expo[index] = ('0' == h[index]) ? 0 : 1;
    } 
  }
  
 
  printf("%s\n", buffer);
  char *h = strchr(buffer,'1');
  h++;
  index = 0;
  while(index < in->mantbits && *h){
    if(*h == '.'){
      h++;
    }
    else{
      in->mantissa[index] = ('0' == *h) ? 0 : 1;
      index++;
      h++;
    }
  }
  
  //HI Brian!
  return 0;
}
\end{lstlisting} This is where I wanted to hit my head against a desk multiple times. Here we do the meat of the math and storing values. I won't go super into 
it because it is very long but I will explain my process. The first thing I did was set the sign because that was found in main. I then passed my input number 
to my binary to decimal converter I wrote for exercise 1. After that came out I made a pointer for the '.' because I wanted to know whether or not I had one 
from the converter. Once I had one pointed at then I could find out how much I needed to shift it. I need to know the shift because I needed the value to know 
how to print my mantissa with a leading 1, and I need the nominal value to calculate the exponent. There are a couple of condition checks for the shift, such as 
what is there isn't a decimal or what happens if only a decimal was input to the converter. Once I found the shift I was able to compute the bias because after 
all the -127 is only for 32 bit representation. Once I had the bias I could compute the binary for exponent with the converter. Now I could write the binary 
into my struct. This gave me a headache because of how the user could input a different number of bit than were calculated by the converter. So there has to be 
a condition for less than or more than enough bits. The same logic goes for mantissa, but it wasn't as bad to write because I can truncate in the mantissa. Last 
was the mantissa. I created a pointer that pointed at the first one it found. This would be my leading one. I then just stored by counting the pointer up until 
it reached the end of the character array. That is the gist of the set function. \newline \newline Main is pretty boring, so boring I won't even copy the code. 
All it did was ask for the inputs and call the functions needed. The reason I made my code this way was because the exercise statement was very vague. I wasn't 
sure if it just wanted IEEE representation, so I made the code print with variable floating point representation. \section{Exercise 6} This code in my opinion 
was a lot easier to write because the algorithm was basically given to us. All we had to do was set up the code so it did math like math is supposed to be done. 
I found a quadratic calculator online and then adjusted a couple of lines to do what we were required to do. I copied the base for a couple reasons. It came 
from an open source code, and I have been told programmers are supposed to be lazy. I did read through and understand the code. I made the inputs doubles so it 
could hold very large inputs. I also set up the if statement that checked to see whether or not we needed to use the second equation provided. This equation was 
needed when the denominator of the first equation was equal to zero. We also needed to make sure that we didn't have overflow or underflow. I did this by 
putting in very weird inputs and checked the roots against my calculator. I couldn't find an input that didn't work. \section{Exercise 7} Last but not least. 
This was a simple series calculator. We have done this before so the premise was almost the same. I studied my Maclaurin series calculated. The first thing I 
did was check to see if the I got my flag. I used the bit of code from the last lab used to search for things on the command line. I then just did the small 
loop to compute the series using the eqaution provided in the exercise. There was also a if else statement to do the calculation for the float flag. Now when 
running the code with the equation we get the wrong answer. The right answer is 6 but we get nearly 100 when using double precision. This is because we are 
compounding an error after each iteration because the floating points aren't computing correctly. \end{document}
