\documentclass{article} \usepackage[utf8]{inputenc} \title{AERE 361 Lab 7} \author{Jordan Reese} \date{Due Date: March, 2nd 2018} \begin{document} \maketitle 
\section{Integrating with Computers} \indent Integration is one of the major pillars of calculus. When first learning integration, we do all the math on paper. 
Our brain can analytically solve the integral. We know reverse product rule and such. Computer on the other hand do not have the complex thinking power a brain 
does. Computer are very fast though. This makes up for having lower thinking capacity. Integrating on a computer can be very useful, but we need to address the 
problem from a different perspective. This perspective is called numerical methods. There are lots of methods, such as Newton-Raphson, and the Gaussian Quad. 
The methods turn calculus into an equation that needs to be iterated. Well computers are great at doing iterations. This lab focused primarily on using C to 
write some integration solvers. We had four methods I will go over, Midpoint, Simpson's $1/3$ and $3/8$, rule and the Gaussian Quad. \section{Midpoint} \indent 
The midpoint technique is fairly simple in nature. We are finding the midpoint of our function and treating that as our height of our rectangles. They theory is 
the section that gets cut off my the function curve will be made up for by the empty space between rectangles. This is a great approximation for linear systems, 
but as we start to traverse into my complex curves, it doesn't hold up very well. The coding behind this one was very straight forward. All I did was put the 
equation for the midpoint method and give it values. The equation is $$\int_{a}^{b} f(x) dx = (b-a)f(\displaystyle \frac{a+b}{2})$$ The time complexity for this 
one is just 1. The equation only needs its input and it will run just once. \section{Simpson Rule} This is another equation based estimation for integrals. It 
comes in 2 variants, the 1/3 and the 3/8. This methods shares the same complexity as midpoint. Once it gets the inputs it wants the equation runs once. For this 
the coding wasn't particularly challenging. Unlike the midpoint method this method will work with all sorts of curves. I was doing some testing and I couldn't 
find one that it failed on. The equations for 1/3 and 3/8 are below with respect to order. $$\int_{a}^{b} f(x) dx = \displaystyle 
\frac{b-a}{6}*[f(a)+4f(\frac{a+b}{2})+f(b)]$$ and $$\int_{a}^{b} f(x) dx = \displaystyle\frac{b-a}{8}*[f(a)+3f(\frac{2a+b}{3})+3f(\frac{a+2b}{3})+f(b)]$$ 
\section{Gaussian Quad} \indent The Gauss Quad is our first iterative method. We will sum to a specified order n. There is a table of w and t values that were 
provided to us by Dr. Mitra. There is a way to find the values, but I am not sure how to do it, nor did I need to. I struggled a little on this one because we 
need to put all the values for t and w into arrays. We then had to index those array inside our loop. I struggle to get all of that figured out. This one also 
has a higher complexity, but it isn't anything too crazy. It loops linearly based on the n input. I also made a terrible terrible mistake. For my arrays I used 
square brackets instead of curly braces because I was fresh from MatLab.
\end{document}
